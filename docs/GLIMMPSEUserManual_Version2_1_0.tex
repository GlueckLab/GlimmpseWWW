%% LyX 2.0.7 created this file.  For more info, see http://www.lyx.org/.
%% Do not edit unless you really know what you are doing.
\documentclass{glimmpse-manual}
\usepackage[latin9]{inputenc}
\setlength{\parskip}{\medskipamount}
\setlength{\parindent}{0pt}
\usepackage{color}
\usepackage{float}
\usepackage{amsmath}
\usepackage{graphicx}
\usepackage[authoryear]{natbib}

\makeatletter

%%%%%%%%%%%%%%%%%%%%%%%%%%%%%% LyX specific LaTeX commands.
%% Because html converters don't know tabularnewline
\providecommand{\tabularnewline}{\\}
%% A simple dot to overcome graphicx limitations
\newcommand{\lyxdot}{.}


%%%%%%%%%%%%%%%%%%%%%%%%%%%%%% Textclass specific LaTeX commands.
 \GLIMMPSEVersion{<enter-GLIMMPSEVersion-in-preamble>}
 \doctitle{<enter-doctitle-in-preamble>}
 \docdate{<enter-docdate-in-preamble>}
 \docauthor{<enter-docauthor-in-preamble>}
 \setlength{\parskip}{0.7ex plus0.1ex minus0.1ex}
 \setlength{\parindent}{0em}

%%%%%%%%%%%%%%%%%%%%%%%%%%%%%% User specified LaTeX commands.

\GLIMMPSEVersion{2.1.0}
\doctitle{GLIMMPSE User Manual}
\docdate{March 2014}
\docauthor{Zacc Coker-Dukowitz, Brandy Ringham, and Sarah Kreidler}

\makeatother

\begin{document}

\section{Introduction}


\subsection{Version Information and Licensing}

This manual describes version 2.1.0 of the GLIMMPSE software. The
manual applies to all 2.1.x versions of GLIMMPSE (e.g. 2.1.0, 2.1.1,
2.1.2, etc.).

GLIMMPSE is released under the \href{http://www.gnu.org/licenses/gpl-2.0.html}{GNU Public License version 2.0}.

The GLIMMPSE program is free software. Users can redistribute it and/or
modify it under the terms of the GNU General Public License as published
by the Free Software Foundation, using either version 2 of the License,
or any later version. This program is distributed in the hope that
it will be useful, but WITHOUT ANY WARRANTY\textemdash{}without even
the implied warranty of MERCHANTABILITY or FITNESS FOR A PARTICULAR
PURPOSE. See the GNU General Public License for more details. 

You should have received a copy of the GNU General Public License
along with this program. If you have not received a copy of the GNU
General Public License and would like one, please write to the Free
Software Foundation, Inc., 51 Franklin Street, Fifth Floor, Boston,
MA 02110-1301, USA.


\subsection{Welcome to GLIMMPSE 2.1.0}

GLIMMPSE 2.1.0 is an open-source online tool for calculating power
and sample size. GLIMMPSE has been designed so that researchers and
scientists with a varying levels of statistical training can have
access to reliable power and sample size calculations. For optimum
usability, GLIMMPSE provides two different modes. In \textit{Guided
Mode} users receive step-by-step guided instructions for entering
data in order to obtain power and sample size outputs. In \textit{Matrix
Mode} users receive less guidance, and are assumed to possess in-depth
statistical training.

GLIMMPSE can compute power or sample size for univariate and multivariate
linear models with Gaussian errors \citet{muller_linear_2006}. GLIMMPSE
supports two main types of study design models: designs with only
fixed predictors, and designs with fixed predictors and a single Gaussian
covariate. The values of a fixed predictor are set as part of the
study design, and are known without appreciable error. In contrast,
Gaussian covariates are not observed until data is collected. Common
designs with only fixed predictors include t-tests, analysis of variance
(ANOVA), and multivariate analysis of variance (MANOVA). Common designs
that control for a covariate include analysis of covariance (ANCOVA)
and multivariate analysis of covariance (MANCOVA).

Details about power calculations for the general linear multivariate
model with Gaussian data and fixed predictors can be found in \citep{muller_practical_1984},
\citep{muller_approximate_1989}, \citep{muller_power_1992}, \citep{muller_statistical_2007},
\citep{muller_linear_2006}, and \citep{muller_statistical_2007}.
Details for fixed predictors with a single Gaussian covariate can
be found in \citep{glueck_adjusting_2003}.

GLIMMPSE utilizes a Java web services architecture \citep{mcgovern_java_2003},
designed to facilitate future support of additional statistical models.
The tool is hosted at \href{http://glimmpse.samplesizeshop.org}.


\subsection{Why GLIMMPSE?}

Other programs, such as POWERLIB, NQuery, and Pass, also calculate
power and sample size. So why use GLIMMPSE?

GLIMMPSE has several advantages over these other programs, because
GLIMMPSE: 
\begin{enumerate}
\item \textbf{Is free}. GLIMMPSE provides free online power and sample size
computing. 
\item \textbf{Is user friendly}. In both \textit{Guided Mode} and \textit{Matrix
Mode} GLIMMPSE provides a step-by-step interface to assist researchers
in producing accurate power and sample size calculations. 
\item Calculates power and sample size for any univariate or multivariate
test for the general linear multivariate model, assuming fixed predictors. 
\item Produces confidence intervals on power estimates for designs with
fixed predictors. 
\item Produces power and sample size calculations for designs with a single
Gaussian covariate. 
\item Supports designs with unequal group sizes, and complicated covariance
structures. 
\item Creates basic power curves. 
\end{enumerate}

\section{Using GLIMMPSE}


\subsection{When to Use GLIMMPSE}

GLIMMPSE is a tool researchers and scientists can use to calculate
reliable values for power and sample size. GLIMMPSE calculates power
or sample size for designs with normally distributed outcomes, and
for a variety of multilevel and longitudinal studies. GLIMMPSE can
calculate power and sample size for common statistical tests and models
including:
\begin{itemize}
\item One sample t-test 
\item Paired t-test 
\item Two sample t-test 
\item Analysis of variance (ANOVA) 
\item Analysis of covariance (ANCOVA) 
\item Repeated measures analysis of variance 
\item Multivariate analysis of variance (MANOVA) 
\item Multivariate analysis of covariance (MANCOVA) 
\end{itemize}

\subsection{How to Use GLIMMPSE}


\subsubsection{Initiating the GLIMMPSE Wizard}

GLIMMPSE can be accessed with a standard web browser at \href{http://glimmpse.samplesizeshop.org/}.
The GLIMMPSE start screen is shown in Figure \ref{fig:glimmpseModeSelect}.
GLIMMPSE has been tested in Internet Explorer 8 \citep{microsoft_internet_2010},
Mozilla Firefox 13.0.1 \citep{mozilla_firefox_2011}, Google Chrome
23.0.1271.95 \citep{google_chrome_2011} and Safari 5.0.3 \citep{apple_safari_2010}.

\begin{figure}[H]
\begin{centering}
\includegraphics[width=6in]{Figures/modeSelect} 
\par\end{centering}

\caption{GLIMMPSE Start Screen}


\label{fig:glimmpseModeSelect} 
\end{figure}



\subsubsection{Choosing Between Guided Mode and Matrix Mode}

The GLIMMPSE start screen presents three options: \textit{Guided Mode},
\textit{Matrix Mode}, and \textit{Upload a Study Design}.

In \emph{Guided Mode} users receive step-by-step guided instructions
when entering inputs for power or sample size calculations. To choose
Guided Mode, click the Guided Study Design box.

In \emph{Matrix Mode} users receive less guidance, and are assumed
to possess in-depth statistical training. \emph{Matrix Mode} allows
direct input of all matrices required for a power or sample size calculation.
To choose Matrix Mode, click the Matrix Study Design box.

If the user has a study design saved from a previous GLIMMPSE session,
the user may upload it by clicking the \emph{Upload a Study Design}
box. GLIMMPSE will open the saved study design and allow the user
to continue the power or sample size analysis.


\subsection{Basic Navigation for GLIMMPSE in Both \textit{Guided Mode} and \textit{Matrix
Mode}}

Once a mode of entry has been chosen, the steps required for GLIMMPSE
to calculate power are listed as tabs on the left side of the \textit{Introduction}
screen. A white background indicates a tab as active, and a blue background
designates a tab as inactive. Blue text designates a page within a
tab as active, and gray designates a page as inactive. Only one page
within one tab can be active at a time.

On the upper right of any screen in GLIMMPSE is a menu of options
enabling users to save their study design by clicking \includegraphics[height=11pt]{Figures/SaveButton},
consult the help library by clicking \includegraphics[height=11pt]{Figures/HelpButton},
or cancel without saving and return to the \textit{Start Your Study
Design} screen by clicking \includegraphics[height=11pt]{Figures/ClearButton}.

Each section is broken into one or more sub-sections with the title
in bold at the top of the page. Each screen contains instructions
and/or areas for user inputs. Users navigate by clicking on the section
titles and sub-sections in the left navigation panel.

\begin{figure}[H]
\begin{centering}
\includegraphics[width=6in]{Figures/MainWizard} 
\par\end{centering}

\caption{Example Start Screen}


\label{fig:mainWizard} 
\end{figure}


Following the \textit{Introduction} screen (Figure \ref{fig:mainWizard}),
GLIMMPSE will prompt the user to enter the details for the power or
sample size calculation. The user may enter the details in any order.
However, some screens cannot be accessed unless the user has completed
information from previous screens. For example, if the user has not
entered information in the \textit{Response Variables} tab, then the
user will not be able to enter information in the \textit{Hypothesis}
tab. If a tab is inaccessible due to missing information in an earlier
screen, it will be indicated by a grey circle with a slash through
it \includegraphics{Figures/GreyCircle}:

\begin{figure}[H]
\begin{centering}
\includegraphics[height=3in]{Figures/LeftNavNotAllowed} 
\par\end{centering}

\caption{More information is required to access the \textit{Hypothesis} screen.}


\label{fig:leftNavNotAllowed} 
\end{figure}


Notice that in Figure \ref{fig:mainWizard} above there are pencil
icons beside the \emph{Desired Power} screen in the active \textit{Start
}tab. The pencil icons indicate screens which require additional information.
Once the user have entered the required information, the pencil will
turn into a green check mark, as shown in Figure \ref{fig:leftNavComplete}.

\begin{figure}[H]
\begin{centering}
\includegraphics[height=3in]{Figures/leftNavComplete} 
\par\end{centering}

\caption{Indication that the \textit{Solving for} screen is complete.}


\label{fig:leftNavComplete} 
\end{figure}


Some screens are optional, and will already have a check mark beside
them.


\subsubsection{Typing Into a Text Box}

Several screens in GLIMMPSE will ask you to specify information by
typing into a text box. To input information in a text box, click
in the text box and type the requested information. To complete the
entry you may press \includegraphics[height=11pt]{\lyxdot \lyxdot /2\lyxdot 0\lyxdot x/Figures/EnterKey}
on your keyboard. 

To delete entries in a list associated with the text box, click \includegraphics[height=11pt]{Figures/DeleteButton}
to delete the entry.

Figure \ref{fig:textBoxes} shows three examples of text box entries
with the text boxes highlighted in blue.

\begin{figure}[H]
\noindent \begin{centering}
\includegraphics[width=4in]{Figures/TextBoxes}
\par\end{centering}

\caption{Examples of text boxes that are used to A) list the desired power
values; B) enter response variables name; and C) specify scale factors
for means.}


\label{fig:textBoxes}
\end{figure}



\subsubsection{Using Drop Down Lists}

When GLIMMPSE requires you to choose from a defined list of options,
these options will be presented in a drop down list. Figure \ref{fig:dropdownLists}
shows an example of a drop down list. To choose an option from a drop
down list, click on the down arrow (see $\left[1\right]$), then select
your choice from the list of options (see $\left[2\right]$).

\begin{figure}[H]
\begin{centering}
\includegraphics[height=2in]{Figures/dropdownLists} 
\par\end{centering}

\caption{Example of a drop down menu.}


\label{fig:dropdownLists} 
\end{figure}



\subsubsection{Radio Buttons and Check Boxes}

In some cases, you must choose from a list of options by selecting
a radio button or checking a box. The radio buttons allow you to select
only one option. The check boxes allow you to select more than one
option. To select an option, click on the radio button or check the
box next to that option. Figure \ref{fig:radioButtonsAndCheckBoxes}
shows an example of a radio button (see A), and a check box (see B).

\begin{figure}[H]
\begin{centering}
\includegraphics[width=6in]{Figures/RadioButtonsAndCheckBoxes} 
\par\end{centering}

\caption{Example of radio buttons and check boxes.}


\label{fig:radioButtonsAndCheckBoxes} 
\end{figure}



\subsubsection{Results Report}

Power results are displayed in a table with each row representing
an individual power calculation. If multiple factors have been specified
in the study design (for example, multiple Type I error rates, variability
scale factors, etc.) then the results table will have multiple rows.
See Table \ref{table:info} below for an example of the information
displayed for a given results report.

Every results report for power contains both calculated and desired
power values. When solving for power, these two values are the same.
When solving for sample size, it may not be possible to achieve the
exact power value specified by the user. In this case, \textit{Target
Power} is the default power value (the power value specified by the
user), and \textit{Power} is the calculated power for the sample size
that best matches the desired power.

A power curve may also be requested, with power on the vertical, or
\emph{Y}, axis and either the regression coefficient scale factor,
covariance scale factor, or total sample size on the horizontal, or
\emph{X}, axis.

Power results can be saved to a comma delimited file so that users
can import the data into other statistical packages. To save the power
results, click \includegraphics[height=25pt]{Figures/SaveResultsButton}
from the \includegraphics[height=11pt]{Figures/SaveButton} menu on
the upper right corner of the screen. For transparency, the matrices
used in the calculations are accessible on the results screen. This
is most useful in \textit{Guided Mode}, where matrix information is
largely hidden from the user. To view the exact matrices used in the
calculations, click \includegraphics[height=35pt]{Figures/ViewMatricesButton}. 

\begin{table}[H]
\begin{centering}
\begin{tabular}[t]{ll}
\hline 
Column Name  & Description\tabularnewline
\hline 
Power  & Calculated power\tabularnewline
Confidence Limits & \textcolor{black}{Lower and Upper limits of the 95\% confidence interval}\tabularnewline
Total Sample Size  & Total number of research participants required to achieve the actual
power\tabularnewline
Target Power & The desired power\tabularnewline
Type I Error Rate & The Type I error value\tabularnewline
Test  & Name of the statistical test\tabularnewline
Means Scale Factor  & Scale factor applied to the $\mathbf{B}$ or $\mathbf{B}_{F}$ matrix\tabularnewline
Variability Scale Factor  & Scale factor applied to the $\mathbf{\Sigma}_{E}$ matrix\tabularnewline
Power Method  & Indicates whether conditional, unconditional, or quantile power was
used\tabularnewline
Quantile  & If the current power method is quantile power, this indicates the
quantile\tabularnewline
 & of the distribution of possible powers. Otherwise, this field is empty.\tabularnewline
\hline 
\end{tabular}
\par\end{centering}

\caption{Information displayed for each power result.}


\label{table:info} 
\end{table}



\subsection{Basic Navigation for \textit{Guided Mode}}


\subsubsection{Entering Predictor Variables\label{sec:Entering-Predictor-Variables}}

In Guided Mode, GLIMMPSE requires the user to enter labels for predictor
variable(s) (also called independent variables) and outcome variable(s)
(also called dependent variables). Figure \ref{fig:enteringPredictors}
shows an example of entering variable labels. To enter a variable
label, type the label into the text box provided (see {[}1{]} in Figure
8). After each entry, press \includegraphics[height=11pt]{\lyxdot \lyxdot /2\lyxdot 0\lyxdot x/Figures/EnterKey}
on the keyboard.

For the predictor variables, GLIMMPSE also asks the user to specify
the categories for each variable. For example, the predictor variable
``gender'' has two categories, ``male'' and ``female.'' To specify
categories associated with a given predictor, select a predictor in
the text box on the left (see {[}2{]}), then enter the category labels
into the text box on the right (see {[}3{]} in Figure 8). After each
entry, press \includegraphics[height=11pt]{\lyxdot \lyxdot /2\lyxdot 0\lyxdot x/Figures/EnterKey}
on the keyboard.

Only category labels associated with the highlighted predictor label
are shown. To delete predictors or category labels, select the unwanted
label and click \includegraphics[height=11pt]{Figures/DeleteButton}.
This removes the label from the list. If the user removes a predictor,
the associated categories are automatically deleted.

\begin{figure}[H]
\noindent \begin{centering}
\includegraphics[scale=0.85]{Figures/Capture}
\par\end{centering}

\caption{Example of entering labels\label{fig:enteringPredictors}}
\end{figure}



\subsection{Basic Navigation for \textit{Matrix Mode}}


\subsubsection{Resizing and Entering Values Into a Matrix}

In Matrix Mode, GLIMMPSE requires the user to define the matrices
for the power calculation. Figure \ref{fig:matrixInput} shows an
example of a matrix template in \textit{Matrix Mode}. Sometimes the
matrix dimensions are pre-determined. If not, the user can set the
matrix dimensions by typing the number of rows into the row text box
(see {[}1{]} in Figure \ref{fig:matrixInput}) and the number of columns
into the column text box (see {[}2{]} in Figure \ref{fig:matrixInput}).
Fill in the elements of the matrix by entering values into the text
boxes within the matrix template (see {[}3{]} in Figure \ref{fig:matrixInput}).

\begin{figure}[H]
\noindent \centering{}\includegraphics{Figures/Capture1} \caption{Example of entering values into a matrix\label{fig:matrixInput}}
\end{figure}



\section{Using \textit{Guided Mode: A Screen-by-Screen Tour}}

In \textit{Guided Mode} users receive step-by-step guided instructions
when entering inputs for calculating power and sample size for use
in study design.


\subsection{Start}


\subsubsection{Introduction}

The \emph{Introduction} screen contains a summary of the steps involved
in the power or sample size analysis.

\begin{figure}[H]
\noindent \begin{centering}
\includegraphics[width=6in]{Figures/IntroScreen}
\par\end{centering}

\caption{Introduction Screen}


\end{figure}



\subsubsection{Solving For?}

The \emph{Solving For?} screen allows the user to select either a
power or sample size calculation.

\begin{figure}[H]
\noindent \begin{centering}
\includegraphics[width=6in]{Figures/SolvingForScreen}
\par\end{centering}

\caption{Solving For? Screen}


\end{figure}


When \emph{power} is selected, the inputs will be used for a power
analysis. The power analysis will produce a value(s) between 0 and
1, representing the probability the study will provide an answer to
the question of interest. When \emph{sample size} is selected, the
inputs will be used to calculate the number of individual sampling
units (also called participants, if referring specifically to people)
needed for the study to achieve the desired power.

If the number of participants is not set, we recommend solving for
sample size in order to obtain the appropriate sample size for achieving
the goals of your study. However, if the sample size is set due to
budgetary or other restrictions, a power calculation will indicate
the probability that your study will provide a definitive answer to
the question of interest. 

On the screen, select \emph{Power} or \emph{Total Sample Size} by
selecting the appropriate radio button.


\subsubsection{Desired Power (if solving for Total Sample Size)}

When solving for sample size, the user must enter the desired power
for the study. Enter the target values as decimals, i.e. 0.95, in
the Power Values box and press \includegraphics[height=11pt]{\lyxdot \lyxdot /2\lyxdot 0\lyxdot x/Figures/EnterKey}
on the keyboard to add the value to the list.

\begin{figure}[H]
\noindent \begin{centering}
\includegraphics[width=6in]{Figures/DesiredPowerScreen}
\par\end{centering}

\caption{Desired Power Screen}


\end{figure}



\subsection{Model}


\subsubsection{Introduction}

This screen provides an introduction to the \textit{Model} section
and defines the concept of an independent sampling unit. The sampling
unit is typically the study participant. For multilevel designs and
cluster randomized trials, the sampling unit may be a group of participants,
such as schools or neighborhoods.

\begin{figure}[H]
\noindent \begin{centering}
\includegraphics[width=6in]{Figures/MeansIntroScreen}
\par\end{centering}

\caption{Sampling Units: Introduction Screen}


\end{figure}



\subsubsection{Clustering}

Clustering is present when research participants are organized into
groups. Often, randomization in a study occurs at the group level
rather than by individual research participants. The \emph{Clustering}
screen allows the user to enter up to three levels of clustering.

An example of clustering would be a study design in which the participants
are students randomly selected from different schools in an area.
In this case, each school would represent a cluster. An example of
subgroups within a cluster would be each classroom within a given
school. 

\begin{figure}[H]
\noindent \begin{centering}
\includegraphics[width=6in]{Figures/ClusteringScreen}
\par\end{centering}

\caption{Clustering Screen}
\end{figure}


\textcolor{black}{If the study design does not include clustering,
simply click on other sub-sections to proceed. }

To add clustering, click the \textit{Add clustering} button. Three
text boxes will appear at the bottom of the screen:

\begin{figure}[H]
\noindent \begin{centering}
\includegraphics[width=6in]{Figures/ClusteringOptions}
\par\end{centering}

\caption{Clustering Options}
\end{figure}


Enter the \textit{Cluster name}, specify the \textit{Number of observations
or sub-clusters within each cluster of this type}, and specify the
\textit{Intra-cluster correlation}. The Intra-cluster correlation
is the expected correlation between pairs of observations within the
cluster.

To add a subgroup to the cluster, click \textit{Add subgroup} and
fill in the information for that subgroup. GLIMMPSE allows one primary
cluster and two subgroups.

\begin{figure}[H]
\noindent \begin{centering}
\includegraphics[width=6in]{Figures/ClusterSubGroupScreen}
\par\end{centering}

\caption{Clustering Sub-Groups}
\end{figure}


Continuing with the above example, the subgroup \textit{Cluster name}
would be ``classroom,'' the \textit{Number of observations} would
be the number of students within each classroom, and the \textit{Intra-cluster
correlation} would be the expected agreement between students within
each classroom.

To remove a subgroup or remove clustering, simply click \textit{Remove
subgroup} or \textit{Clear All}.


\subsubsection{Predictors\label{sec:Predictors}}

Independent sampling units may be randomized to different treatments,
or be classified by characteristics such as gender. The characteristics
divide the sampling units into study groups. The \emph{Predictors}
screen allows the user to define the study groups by specifying \textit{fixed
predictors}. Enter the fixed predictors as described in Section \ref{sec:Entering-Predictor-Variables}.
For one-sample designs with no fixed predictors, leave the table blank.

\begin{figure}[H]
\noindent \begin{centering}
\includegraphics[width=6in]{Figures/PredictorsScreen}
\par\end{centering}

\caption{Predictors Screen}


\end{figure}



\subsubsection{Covariate}

The \emph{Covariate} screen allows the user to control for a single,
normally distributed predictor (also known as a normally distributed
covariate). For example, a scientist may wish to examine the effect
of a drug when controlling for age. In this case, age would be the
covariate. If the study design does not include a normally distributed
predictor, leave the checkbox blank. If the study design does include
a covariate, check the checkbox.

\begin{figure}[H]
\noindent \begin{centering}
\includegraphics[width=6in]{Figures/GaussianPredictorScreen}
\par\end{centering}

\caption{Gaussian Predictor Screen}


\end{figure}



\subsubsection{Response Variables}

The \textit{Response Variables }screen allows the user to specify
the response or dependent variables for the study. For example, if
``expected pain'' is the desired outcome, enter ``expected pain''
in the text box.

\begin{figure}[H]
\noindent \begin{centering}
\includegraphics[width=6in]{Figures/ResponseVariablesScreen}
\par\end{centering}

\caption{Response Variables}
\end{figure}



\subsubsection{Repeated Measures}

The \emph{Repeated Measures} screen allows the user to describe repeated
measures. Repeated measures are present in a study when multiple measurements
are taken on each research participant. An example of repeated measures
would be researchers taking a participant's blood pressure once a
month for six months.

\textcolor{black}{If the design does not have repeated measures, simply
click on other sub-sections to proceed}\textcolor{red}{. }If the design
includes repeated measures, click \textit{Add repeated measures}\textit{\emph{
and fill in the requested information.}}

\textit{Units} is a user-specified description of the repeated measure.
For example, if the repeated measures are taken once every month,
the unit could be ``month.'' Enter a label for the units of the
repeated measure.

Enter the \textit{Type} of unit. For \textit{Numeric} repeated measures,
both the distance and ordering between measurements is meaningful.
Measuring blood pressure every month for 6 months is a numeric repeated
measure. GLIMMPSE will auto-populate an equal distance between repeated
numeric measures. You can change the distance between the measures
by typing into the text boxes. For \textit{Ordinal} repeated measures,
the ordering of the measurements is meaningful, but the distance between
measurements is assumed to be equal. For example, repeated measures
of the participant's heart rate taken in the morning, afternoon, and
evening. For \textit{Categorical} repeated measures, neither the ordering
nor the distance between the measures is meaningful. For example,
repeated measures of breast density using three different instruments,
Device A, B, and C.

\textit{Number of Measurements} allows you to specify the number of
times the repeated measure will be taken. For the blood pressure example,
the \textit{Number of Measurements} would be 6 because blood pressure
was measured every 6 months. For numeric repeated measures, GLIMMPSE
2.0 auto-populates equidistant measurements. To change the distance
between measures, type into the text boxes. For example, if blood
pressure was measured every month for the first three months, then
every other month for the next six months, the user would type 1,
2, 3, 5, 7, 9 into the text boxes.

\begin{figure}[H]
\noindent \begin{centering}
\includegraphics[width=6in]{Figures/RepeatedMeasuresScreen}
\par\end{centering}

\caption{Repeated Measures}
\end{figure}


Nested repeated measures are added via the \emph{Add Level} button.
For example, consider a design in which a participant's blood pressure
is measured every month for six months, and at each visit in three
different positions (for example, standing, sitting, and supine).
The design would include doubly repeated measures with one level for
``month'' and a second nested level for ``position.'' The user
may add up to three levels of repeated measures.

To add a sub-level, click the \textit{Add level}\textit{\emph{ button.
T}}hree more text boxes will appear:

\begin{figure}[H]
\noindent \begin{centering}
\includegraphics[width=6in]{Figures/RepeatedMeasuresSub-Dimensions}
\par\end{centering}

\caption{Repeated Measures: Add Level}
\end{figure}



\subsubsection{Relative Group Sizes (if solving for Power)}

For designs with multiple study groups (see Section \ref{sec:Predictors}),
the user may specify equal or unequal group sizes. On the Relative
Group Sizes screen, the user can select the relative sizes of each
group by selecting a value from the drop down list.

For example, consider a design with males and females, randomized
to receive either an active drug or a placebo. For equal group sizes,
a ``1'' should be selected for each drop down list as shown in Figure
\ref{fig:Relative-Group-Sizes}. However, if there were twice as many
males receiving the drug compared to females receiving the drug, the
user would select a ``2'' for the male + drug group, and ``1''
for the remaining groups. 

\begin{figure}[H]
\noindent \begin{centering}
\includegraphics[width=6in]{Figures/RelativeGroupSizesScreen}
\par\end{centering}

\caption{Relative Group Sizes Screen\label{fig:Relative-Group-Sizes}}
\end{figure}



\subsubsection{Smallest Group Size}

When solving for power, the user specifies the total sample size for
the design by the relative group sizes and the smallest group size.
On the \emph{Smallest Group Size} screen, the user may enter one or
more values describing the number of participants in the smallest
group.

For example, consider a design with a treatment and a placebo group,
in which three times as many participants receive the treatment compared
to the placebo. With a smallest group size of 20, 30, or 40, the total
sample size for the design would be 80 (i.e. 60 treated participants,
20 with placebo), 120, and 160 participants respectively.

\begin{figure}[H]
\noindent \begin{centering}
\includegraphics[width=6in]{Figures/SmallestGroupSize}
\par\end{centering}

\caption{Smallest Group Size Screen}
\end{figure}



\subsection{Hypothesis}


\subsubsection{Introduction}

This screen provides an introduction to the \textit{Hypothesis} section. 

\begin{figure}[H]
\noindent \begin{centering}
\includegraphics[width=6in]{Figures/HypothesisIntro}
\par\end{centering}

\caption{Hypothesis Introduction Screen}


\end{figure}



\subsubsection{Hypothesis}

The \textit{Hypothesis} screen allows the user to select the primary
hypothesis of interest. The user first selects the type of hypothesis
by clicking the drop down list. Additional information will be requested
depending on the type of hypothesis.

\begin{figure}[H]
\noindent \begin{centering}
\includegraphics[width=6in]{Figures/Hypothesis}
\par\end{centering}

\caption{Hypothesis Screen}


\end{figure}


A \textit{Grand Mean} hypothesis compares the overall mean response
in a sample of participants against a known value. For example, an
investigator may wish to determine if body mass index values for participants
in a particular state differs from the United States national average.
After selecting the \emph{Grand mean} radio button, the user will
be prompted to enter the known mean value for each response variable. 

A \emph{Main effect} hypothesis tests for the effect of a single predictor
variable averaged across all other factors. For example, testing whether
responses of participants in the treatment group differ on average
from participant responses in a placebo group is a common main effect
hypothesis. After selecting the \emph{Main effect} radio button, the
user will be prompted to select one predictor of interest by selecting
the appropriate radio button. 

A \emph{Trend} hypothesis tests whether the effect of a single predictor
follows a particular polynomial pattern, such as a linear or quadratic
trend, across different levels of the predictor. After selecting the
\emph{Trend} radio button, the user will be prompted to select one
predictor of interest. In addition, the user may select from six possible
trends: no trend, change from baseline, linear trend, quadratic trend,
cubic trend, or all polynomial trends.

An \emph{Interaction} hypothesis tests whether the effect of one predictor
changes depending on the value of one or more additional predictors.
An interaction test also can be interpreted as a test of differences,
as well as a test of parallel trajectories of response. For example,
testing whether the effect of a cholesterol lowering medication on
total serum cholesterol differs depending on the participant's gender
is an example of an interaction hypothesis. After selecting the \emph{Interaction}
radio button, the user will be prompted to select one or more factors
of interest by clicking the appropriate check boxes. In addition,
the user may specify a trend for given factor by clicking the \emph{Edit
trend} button. 


\subsubsection{Statistical Test}

The \emph{Statistical Test} screen allows the user to select one or
more statistical tests for the power or sample size calculations.
A tutorial providing guidelines for selecting a test is available
from the GLIMMPSE Tutorials page at \url{http://samplesizeshop.org/education/tutorials}.
Select the statistical test(s) you wish to use by clicking one or
more check boxes.

\begin{figure}[H]
\noindent \begin{centering}
\includegraphics[width=6in]{Figures/StatisticalTests}
\par\end{centering}

\caption{Statistical Tests}
\end{figure}



\subsubsection{Type I Error}

Enter the target values for Type I Error as decimals (i.e. 0.05) in
the Type I Error Values box. The user may specify up to five Type
I Error values.

\begin{figure}[H]
\noindent \begin{centering}
\includegraphics[width=6in]{Figures/TypeIError}
\par\end{centering}

\caption{Type I Error}
\end{figure}



\subsection{Means}


\subsubsection{Introduction}

This screen provides an introduction to the \textit{Means} section. 

\begin{figure}[H]
\noindent \begin{centering}
\includegraphics[width=6in]{Figures/MeansIntro}
\par\end{centering}

\caption{Means Introduction Screen}


\end{figure}



\subsubsection{Means}

The \emph{Means} screen allows the user to enter the expected mean
value for the experiment. Expected mean values are typically drawn
from the literature or from pilot data. Differences between the entered
means typically represent the smallest clinically relevant difference.
The table should contain at least one value that is non-zero.

\begin{figure}[H]
\noindent \begin{centering}
\includegraphics[width=6in]{Figures/Means}
\par\end{centering}

\caption{Means Screen}


\end{figure}


For designs with repeated measures, the user may enter means at each
time (place, etc.). 


\subsubsection{Scale Factors}

The\textit{ Scale Factors} screen for Means allows the user to compute
power or sample size for the means as specified. For example, entering
the scale factors 0.5, 1, and 2 would compute power for the the mean
values divided by 2, the mean values as specified, and the mean values
multiplied by 2. Type each value into the text box and press \includegraphics[height=11pt]{\lyxdot \lyxdot /2\lyxdot 0\lyxdot x/Figures/EnterKey}
on your keyboard. To remove an item, click the \includegraphics{Figures/DeleteButton}
next to the item.

\begin{figure}[H]
\noindent \begin{centering}
\includegraphics[width=6in]{Figures/ScaleFactorsForMeans}
\par\end{centering}

\caption{Scale Factors for Means Screen}


\end{figure}



\subsection{Variability}


\subsubsection{Introduction}

This screen provides an introduction to the \textit{Variability} section. 

\noindent \begin{center}
\begin{figure}[H]
\noindent \begin{centering}
\includegraphics[width=6in]{Figures/VariabilityIntro}
\par\end{centering}

\caption{Variability Introduction Screen}
\end{figure}

\par\end{center}


\subsubsection{Within Participant Variability}

For a given participant, responses may vary across repeated measurements
and for different response variables. The amount of variability can
dramatically impact power and sample size. The \textit{Within Participant
Variability} screen allows the user to describe the variability he
or she expects to observe for each within participant factor and response
variable.

Separate tabs are presented for each ``source'' of correlation in
the design. The \emph{Responses} tab allows the user to specify the
standard deviations of the response variables and any correlation
between them. If repeated measures are present, a single tab will
be presented for each level of repeated measures. Figure \ref{fig:Within-Participant-Variability}
shows an example design in which blood pressure is measured once a
month for six months. GLIMMPSE will automatically combine the sources
of correlation into a final covariance matrix.

\begin{figure}[H]
\noindent \begin{centering}
\includegraphics[width=6in]{Figures/WithinParticipantVariability}
\par\end{centering}

\caption{Within Participant Variability\label{fig:Within-Participant-Variability}}


\end{figure}



\subsubsection{Scale Factors}

While GLIMMPSE requests standard deviations, it actually computes
variances when it conducts the power or sample size calculations.
There may be considerable uncertainty about what standard deviation
or variance value to use. To account for this uncertainty, it is common
to calculate power or sample size using alternative values for variability.
\textcolor{black}{Scale factors allow you to consider alternative
values for variability by scaling the calculated covariance matrix.
For example, entering the scale factors 0.5, 1, and 2 would compute
power for the covariance matrix divided by 2, the covariance matrix
as entered, and the covariance matrix multiplied by 2. }Type each
value into the text box and press \includegraphics[height=11pt]{\lyxdot \lyxdot /2\lyxdot 0\lyxdot x/Figures/EnterKey}
on your keyboard. To remove an item, click the \includegraphics{Figures/DeleteButton}
next to the item.

\begin{figure}[H]
\noindent \begin{centering}
\includegraphics[width=6in]{Figures/ScaleFactorsForVariability}
\par\end{centering}

\noindent \begin{centering}
\caption{Scale Factors for Variability Screen}

\par\end{centering}

\end{figure}



\subsection{Options}

This screen provides an introduction to the Options section.

\begin{figure}[H]
\noindent \begin{centering}
\includegraphics[width=6in]{Figures/Options}
\par\end{centering}

\caption{Options}


\end{figure}



\subsubsection{Power Calculation Method}

For designs with a baseline covariate, two different methods are available
to calculate power: quantile and unconditional power. For theoretical
details, please see \citet{glueck_adjusting_2003}. Select the power
methods by clicking the check boxes. If quantile power is selected,
the user must also specify one or more quantile values. For example,
median power would be obtained by selecting \emph{Quantile} power
and entering ``0.5'' in the quantile list box. 

\begin{figure}[H]
\noindent \begin{centering}
\includegraphics[width=6in]{Figures/PowerMethods}
\par\end{centering}

\caption{Statistical Tests}
\end{figure}



\subsubsection{Confidence Intervals}

Power analysis involves some uncertainty in the choices for means
and variability. Therefore, the \emph{Confidence Intervals} screen
allows the user to request confidence intervals on the power results.
To include confidence intervals, uncheck the checkbox. The information
on the confidence interval screen describes the data set (or publication)
from which the choices for means and variances were obtained. For
example, if a scientist were calculating power based on the means
and variances obtained from pilot data, the scientist would enter
information describing the pilot data set. The following information
is required:

The \emph{Assumptions} section allows the user to indicate if he or
she is uncertain about the variance, but reasonably certain of the
mean values, or uncertain of both the means and variance. 

The \emph{Upper and lower tail probabilities} define the width of
the confidence interval. For example, a centered 95\% confidence interval
would have both upper and lower tail probabilities of 0.025.

The \emph{Total sample size} value indicates the number of independent
sampling units in the pilot data set (or publication).

The \emph{Rank of the design matrix} describes a property of the predictor
matrix used in the pilot data set. Please see \citet{muller_linear_2006}
for details about matrix rank.

\begin{figure}[H]
\noindent \begin{centering}
\includegraphics[width=6in]{Figures/ConfidenceIntervals}
\par\end{centering}

\caption{Confidence Intervals}


\end{figure}



\subsubsection{Power Curve Options}

The \emph{Power Curve Options} screen allows the user to create a
power curve. A power curve describes the change in power (Y axis of
the power curve) relative to the total sample size, regression coefficient
scale factor, or the variability scale factor (all options for the
X axis of the power curve).

To create a power curve, the user must 1) uncheck the check box, 2)
select the value to appear on the horizontal axis, and 3) add one
or more data series.

Depending on the study design, the user may request a large number
of power or sample size values in a single request. A data series
is defined by selecting a subset of the power or sample size values.
The user creates a data series by selecting values for several study
design variables and clicking the checkbox. A data series will be
displayed as a single line on the power curve plot, as shown in Figure
.

\begin{figure}[H]
\noindent \begin{centering}
\includegraphics[width=6in]{Figures/PowerCurveMatrixMode} 
\par\end{centering}

\caption{Power Curve}
\end{figure}


Note that the \emph{Power Curve Options} screen is the final screen
in the GLIMMPSE wizard. \textcolor{black}{If the study design is not
complete, the Calculate button will be disabled}\textcolor{red}{{} \includegraphics[height=11pt]{Figures/CalculateButtonDisabled}}\textcolor{black}{.
If you click on the Calculate button while it is disabled, a modal
will appear on the screen and list the missing inputs, as shown in
Figure \ref{fig:Incomplete-Study-Design}.}

\begin{figure}[H]
\noindent \begin{centering}
\includegraphics[scale=0.6]{Figures/IncompleteStudyDesignPopUp}
\par\end{centering}

\caption{Incomplete Study Design Modal \label{fig:Incomplete-Study-Design}}
\end{figure}



\subsection{Calculate}

When sufficient information for your power or sample size calculation
has been entered, the \textit{Calculate} button will be highlighted
green. Click \includegraphics[height=11pt]{Figures/CalculateButtonEnabled}
to receive the results of your power analysis. Example results are
shown in Figures \ref{fig:Results-Table} and \ref{fig:Results-Plot}.
For detailed information regarding the Power Results table, refer
to Table \ref{table:info}. When calculating sample size for a clustered
design, Power Results provide total sample size, with a breakdown
of sample size per unit, as shown in Figure \ref{fig:Total-Sample-Size-ClusteredDesign}. 

\begin{figure}[H]
\noindent \begin{centering}
\includegraphics[scale=0.6]{Figures/Results}
\par\end{centering}

\caption{Results Table\label{fig:Results-Table}}
\end{figure}


\begin{figure}[H]
\noindent \begin{centering}
\includegraphics[scale=0.6]{Figures/SampleSizeClustererdDesign}
\par\end{centering}

\caption{Total Sample Size for Clustered Design \label{fig:Total-Sample-Size-ClusteredDesign}}
\end{figure}


\begin{figure}[H]
\noindent \begin{centering}
\includegraphics[scale=0.6]{Figures/PowerCurvePlot}
\par\end{centering}

\caption{Results Plot \label{fig:Results-Plot}}


\end{figure}



\section{\textit{Matrix Mode} Screen-by-Screen Tour}

\textit{Matrix Mode} allows direct input of all matrices required
for a power calculation. In \textit{Matrix Mode} users receive less
guidance than in \textit{Guided Mode}, and are assumed to possess
in-depth statistical training. 


\subsection{Start}


\subsubsection{Introduction}

The \emph{Introduction} screen shown in Figure \ref{fig:Introduction-Screen-for-Matrix-Mode}
briefly describes the required matrix inputs for the power or sample
size calculation.

\begin{figure}[H]
\noindent \begin{centering}
\includegraphics[width=6in]{Figures/MatrixIntro}
\par\end{centering}

\caption{Introduction Screen for \emph{Matrix Mode \label{fig:Introduction-Screen-for-Matrix-Mode}}}
\end{figure}



\subsubsection{Solving For?}

The \emph{Solving For?} screen shown in Figure \ref{fig:Solving-For?-Screen-MatrixMode}
allows the user to select either a power or sample size calculation.

\begin{figure}[H]
\noindent \begin{centering}
\includegraphics[width=6in]{Figures/MatrixSolvingFor}
\par\end{centering}

\caption{Solving For? Screen \label{fig:Solving-For?-Screen-MatrixMode}}
\end{figure}


When \emph{Power} is selected, the inputs will be used for a power
analysis. The power analysis will produce a value(s) between 0 and
1, representing the probability the study will answer the question
of interest. When \emph{Total Sample Size} is selected, the inputs
will be used to calculate the number of individual sampling units
(also called participants, if referring specifically to people) needed
for the study to achieve the desired power.

If the number of participants is not set, we recommend solving for
sample size in order to obtain the appropriate sample size for achieving
the goals of your study. However, if sample size is set due to budgetary
or other restrictions, a power calculation will indicate the probability
that the study will provide a definitive answer to the question of
interest. 

On the screen, select \emph{Power} or \emph{Total Sample Size} by
selecting the appropriate radio button.


\subsubsection{Desired Power (if solving for Total Sample Size)}

When solving for sample size, the user must enter the desired power
for the study. Enter the target values as decimals (i.e. 0.95) in
the Power Values box shown in Figure \ref{fig:Desired-Power-Screen-MatrixMode}
and press \includegraphics[height=11pt]{\lyxdot \lyxdot /2\lyxdot 0\lyxdot x/Figures/EnterKey}
on your keyboard to add the value to the list.

\begin{figure}[H]
\noindent \begin{centering}
\includegraphics[width=6in]{Figures/MatrixDesiredPower}
\par\end{centering}

\caption{Desired Power Screen \label{fig:Desired-Power-Screen-MatrixMode}}
\end{figure}



\subsection{Design}


\subsubsection{Design Essence}

In the \textit{Design} section, the user will define the composition
of the study by specifying the number of groups, how subjects are
divided into groups, the size of each group, and whether the design
will include a Gaussian covariate.

\textit{The Design Essence Matrix}

In the general linear multivariate model with fixed predictors, $\boldsymbol{Y}=\boldsymbol{XB}+\boldsymbol{E}$,
the $\boldsymbol{X}$ matrix represents the study design. The same
is true for $\boldsymbol{F}$ in the general linear multivariate model
with fixed predictors and a Gaussian predictor \citep{glueck_adjusting_2003}.
For simplicity, we will only discuss $\boldsymbol{X}$ (since the
instructions do not change for $\boldsymbol{F}$). In data analysis,
the $\boldsymbol{X}$ matrix would contain a single row for each subject.
Since power analysis does not include actual data, the design ``essence''
matrix \citep{muller_linear_2006} is a version of the $\boldsymbol{X}$
matrix that contains a single row for each unique combination of predictors
in the study design. Note that the essence matrix specifies only the
fixed, or categorical, predictors in the study design.

For example, consider a 2-factor ANOVA design with 2 levels per factor,
3 subjects per group, and a cell means coding. In data analysis, the
design matrix and corresponding essence matrix would be:

\[
\boldsymbol{X}=\begin{bmatrix}1 & 0 & 1 & 0\\
1 & 0 & 1 & 0\\
1 & 0 & 1 & 0\\
1 & 0 & 0 & 1\\
1 & 0 & 0 & 1\\
1 & 0 & 0 & 1\\
0 & 1 & 1 & 0\\
0 & 1 & 1 & 0\\
0 & 1 & 1 & 0\\
0 & 1 & 0 & 1\\
0 & 1 & 0 & 1\\
0 & 1 & 0 & 1
\end{bmatrix}\implies\text{Es}\left(\boldsymbol{X}\right)=\begin{bmatrix}1 & 0 & 1 & 0\\
1 & 0 & 0 & 1\\
0 & 1 & 1 & 0\\
0 & 1 & 0 & 1
\end{bmatrix}
\]


GLIMMPSE requires that the design coding is full rank. Unequal group
sizes may be coded by replicating a row to reflect the relative sizes
of the groups.

After entering the desired dimensions for the matrix in the row and
column dimension text boxes, click anywhere on the screen for the
matrix to be resized. Type in the matrix text boxes shown in Figure
\ref{fig:Type-I-Error-MatrixMode} to enter the matrix information.

\begin{figure}[H]
\noindent \begin{centering}
\includegraphics[width=6in]{Figures/XEssence}
\par\end{centering}

\caption{Type I Error \label{fig:Type-I-Error-MatrixMode}}
\end{figure}



\subsubsection{Covariate}

Currently, GLIMMPSE only performs power calculations for hypotheses
about fixed predictor variables. However, a single, continuous, normally
distributed predictor variable may be included in the analysis.

To include such a predictor, click the checkbox next to \textit{Control
for a single, normally distributed Gaussian predictor} at the bottom
of the screen, shown in Figure \ref{fig:Gaussian-Predictor-Screen}.

\begin{figure}[H]
\noindent \begin{centering}
\includegraphics[width=6in]{Figures/MatrixCovariate}
\par\end{centering}

\caption{Gaussian Predictor Screen \label{fig:Gaussian-Predictor-Screen}}
\end{figure}



\subsubsection{Smallest Group Size}

When solving for power, the user specifies the total sample size for
the design by the relative number of repeated rows in the design essence
matrix, and the smallest group size. On the \emph{Smallest Group Size}
screen, the user may enter one or more values describing the number
of participants in the smallest group.

To enter one or more per group sample size, type the sample size in
the\textit{ Per Group Sample Size} box, shown in Figure \ref{fig:Smallest-Group-Size-MatrixMode}.
After each entry, press \includegraphics[height=11pt]{\lyxdot \lyxdot /2\lyxdot 0\lyxdot x/Figures/EnterKey}
on your keyboard. To delete a value, select the unwanted value and
click \includegraphics[height=11pt]{Figures/DeleteButton} to remove
the value from the list.

\begin{figure}[H]
\noindent \begin{centering}
\includegraphics[width=6in]{Figures/MatrixSmallestGroupSize}
\par\end{centering}

\caption{Smallest Group Size Screen \label{fig:Smallest-Group-Size-MatrixMode}}
\end{figure}



\subsection{Coefficients}


\subsubsection{Beta Coefficients: B Matrix}

This section requires the user to enter choices for values for the
hypothesis test, $\boldsymbol{\Theta}=\boldsymbol{CBU}$.

In the general linear multivariate model with fixed predictors, $\boldsymbol{Y}=\boldsymbol{XB}+\boldsymbol{E}$,
the \textbf{$\boldsymbol{B}$} matrix represents the proposed relationship
between the predictor variables, $\boldsymbol{X}$, and the outcome
variables, $\boldsymbol{Y}$. The same is true for $\boldsymbol{B}_{F}$
in the General Linear Multivariate Model with Fixed Predictors and
a Gaussian Predictor. For simplicity, we will only discuss $\boldsymbol{B}$
(since the instructions do not change for $\boldsymbol{B}_{F}$).
To calculate power, enter values for the regression coefficients for
each unique combination of predictors in the study design. The row
dimension of $\boldsymbol{B}$ is determined by the number of columns
in the essence matrix. Change the column dimension of $\boldsymbol{B}$
to match the intended number of outcomes in the study, or the columns
of $\boldsymbol{Y}$ in the general linear multivariate model with
fixed predictors regression equation.

For example, an investigator may want to compare vitamin D and calcium
levels of children who live in three different regions: urban, suburban,
and rural. The $\boldsymbol{B}$ matrix would have three pre-specified
rows, one for each region, and two columns, one for vitamin D and
one for calcium. To calculate power, the investigator must enter the
expected mean vitamin D (column 1) and calcium (column 2) levels of
the children in the rural (row 1), suburban (row 2), and urban (row
3) regions. The investigator may choose $\boldsymbol{B}=\begin{bmatrix}120 & 4\\
60 & 8\\
45 & 10
\end{bmatrix}$ as shown below:

\begin{figure}[H]
\noindent \begin{centering}
\includegraphics[width=6in]{Figures/BetaMatrix}
\par\end{centering}

\caption{Beta Matrix \label{fig:Beta-Matrix-MatrixMode}}
\end{figure}


Enter the number of columns, or the number of outcomes in the study,
in the column text box (right) in the matrix, as shown in Figure \ref{fig:Beta-Matrix-MatrixMode}.
Enter proposed values of the $\boldsymbol{B}$ coefficients in their
corresponding text boxes in the matrix.


\subsubsection{Beta Scale Factors}

GLIMMPSE allows users to specify scale factors for the $\boldsymbol{B}$
matrix in order to generate power or sample size values for different
coefficient values. Since power is based on proposed regression coefficients,
it is common to calculate power for the proposed value, as well as
alternative values such as half and twice the proposed value.

One or more scale factors for the $\boldsymbol{B}$ matrix may be
specified for inclusion in the power calculation. For example, to
calculate power for regression coefficients that are half the values
in your $\boldsymbol{B}$ matrix, enter $0.5$. To use the exact $\boldsymbol{B}$
matrix specified, enter $1$. To delete a value, select the unwanted
value and click \includegraphics[height=11pt]{Figures/DeleteButton}
to remove the value from the list, as shown in Figure \ref{fig:Beta-Scale-Factors-MatrixMode}.

\begin{figure}[H]
\noindent \begin{centering}
\includegraphics[width=6in]{Figures/Betascale}
\par\end{centering}

\caption{Beta Scale Factors \label{fig:Beta-Scale-Factors-MatrixMode} }
\end{figure}



\subsection{Hypothesis}

In this section, the user defines the contrast matrices in the study.
The contrast matrices, $\boldsymbol{C}$ and $\boldsymbol{U}$, consist
of the hypotheses to be tested. They are used to calculate the expected
hypothesis matrix, $\boldsymbol{\Theta}=\boldsymbol{CBU}$.


\subsubsection{Between-Participant Contrast}

The $\boldsymbol{C}$ matrix consists of the between-participant contrasts.
The between-participant contrasts test hypotheses between independent
sampling units. The number of rows in the $\boldsymbol{C}$ matrix
represent the degrees of freedom for the hypothesis test. For example,
suppose an investigator wants to compare the average final exam test
scores of students in class A and class B. The contrast matrix would
be $\boldsymbol{C}=\begin{bmatrix}1 & -1\end{bmatrix}$. When multiplied
by $\boldsymbol{B}$, this becomes the difference in the proposed
average test scores between class A and class B.

Enter the number of rows/number of contrasts in the study, in the
row text box (left) under $\boldsymbol{C}$ Matrix to resize the blank
matrix. Fill in the contrasts you wish to test in the matrix, as shown
in Figure \ref{fig:Between-participant-Contrast-Matrix-MatrixMode}. 

The number of rows in the $\boldsymbol{C}$ matrix cannot exceed the
number of rows in the essence matrix minus 1. In addition, the $\boldsymbol{C}$
matrix must conform to the $\boldsymbol{B}$ matrix, so the number
of columns cannot be adjusted on this screen.

\begin{figure}[H]
\noindent \begin{centering}
\includegraphics[width=6in]{Figures/CMatrix}
\par\end{centering}

\caption{Between-participant Contrast Matrix \label{fig:Between-participant-Contrast-Matrix-MatrixMode}}


\end{figure}



\subsubsection{Within-Participant Contrast}

The $\boldsymbol{U}$ matrix consists of the within-participants contrasts.
The within-participants contrasts are the hypotheses that compare
measurements on the same independent sampling unit.

The $\boldsymbol{U}$ matrix is most useful for multivariate designs
and repeated measures. For example, suppose an investigator wants
to examine whether student test scores improve from their midterm
exams to their final exams. The investigator would have two measurements
per student, one for the midterm and one for the final. The within-participant
contrast matrix would be $\boldsymbol{U}=\begin{bmatrix}1 & -1\end{bmatrix}$.
The matrix contrasts two different test scores, the midterm and the
final, for the same student. 

Enter the number of columns, or the number of within-subject contrasts,
in the study, in the column text box (right). Fill in the contrasts
in the matrix, as shown in Figure \ref{fig:Within-participant-Contrast-Matrix-MatrixMode}.

The $\boldsymbol{U}$ matrix must conform to the $\boldsymbol{B}$
matrix, so the number of rows cannot be adjusted on this screen.

\begin{figure}[H]
\noindent \begin{centering}
\includegraphics[width=6in]{Figures/UMatrix}
\par\end{centering}

\caption{Within-participant Contrast Matrix \label{fig:Within-participant-Contrast-Matrix-MatrixMode}}


\end{figure}



\subsubsection{Null Hypothesis}

The null hypothesis matrix, $\boldsymbol{\Theta}_{0}$, represents
the test values the user expects to observe when the null hypothesis
is true. When performing a power analysis, the values for the hypothesis
tests are calculated as $\boldsymbol{\Theta}$ = $\boldsymbol{CBU}$,
and then compared against $\boldsymbol{\Theta}_{0}$. Commonly, $\boldsymbol{\Theta}_{0}$
is a matrix of zeroes.

For example, suppose an investigator wants to compare resting metabolic
rate between subjects with HIV lipoatrophy, subjects with HIV only,
and healthy controls. The null hypothesis of no difference between
the three groups is $\boldsymbol{\Theta}_{0}=\begin{bmatrix}0\\
0
\end{bmatrix}$, appearing as shown in Figure \ref{fig:Null-Hypothesis-Matrix}.

\begin{figure}[H]
\noindent \begin{centering}
\includegraphics[width=6in]{Figures/ThetaMatrix}
\par\end{centering}

\caption{Null Hypothesis Matrix \label{fig:Null-Hypothesis-Matrix}}


\end{figure}


Sometimes, however, the null hypothesis is based on previous studies
or clinical experience. For example, suppose an investigator wants
to compare foal birth weight between dams who are given feed formula
A, feed formula B, and standard feed. In order to be cost effective,
the new feed formulas must improve foal birth weight by more than
7 kg. The null hypothesis, then is $\boldsymbol{\Theta}_{0}=\begin{bmatrix}7\\
7
\end{bmatrix}$, appearing as shown in Figure \ref{fig:Non-zero-Null-Hypothesis}.

\begin{figure}[H]
\noindent \begin{centering}
\includegraphics[width=6in]{Figures/Theta2}
\par\end{centering}

\caption{Non-zero Null Hypothesis Matrix \label{fig:Non-zero-Null-Hypothesis}}


\end{figure}


$\boldsymbol{\Theta}_{0}$ has the same number of rows as $\boldsymbol{C}$,
and the same number of columns as $\boldsymbol{U}$. Therefore, its
size cannot be adjusted on this screen. The user need only enter the
matrix cell values.


\subsubsection{Statistical Tests}

The \emph{Statistical Tests} screen allows the user to select one
or more statistical tests for the power or sample size calculations.
A tutorial providing guidelines for selecting a test is available
from the GLIMMPSE Tutorials page at \url{http://samplesizeshop.org/education/tutorials}.
Full theoretical details are available in \citet{muller_linear_2006}.
Select the statistical test(s) by clicking one or more check boxes.
For designs with a Gaussian covariate, only the Hotelling-Lawley trace
and the Univariate Approach to Repeated Measures are valid.

\begin{figure}[H]
\noindent \begin{centering}
\includegraphics[width=6in]{Figures/MatrixTests}
\par\end{centering}

\caption{Statistical Tests}
\end{figure}



\subsubsection{Type I Error}

Enter the target values for Type I Error as decimals (i.e. 0.05) in
the Type I Error Values box. The user may specify up to five Type
I Error values.

\begin{figure}[H]
\noindent \begin{centering}
\includegraphics[width=6in]{Figures/MatrixTypeIError}
\par\end{centering}

\caption{Type I Error}
\end{figure}



\subsection{Variability}

Variability describes how much measurements differ from each other.
In this section, the user defines the covariance of errors and covariance
related to the Gaussian covariate. 


\subsubsection{Error Covariance}

For each independent sampling unit, $\boldsymbol{\Sigma}_{e}$ is
the covariance of the random errors conditional on the values of the
fixed predictors. The \emph{Error Covariance} screen allows the user
to define $\boldsymbol{\Sigma}_{e}$ by directly entering the covariance
matrix values, as shown in Figure \ref{fig:Error-Covariance}. To
ensure conformance with the $\boldsymbol{B}$ and $\boldsymbol{U}$
matrices, the dimensions of the $\boldsymbol{\Sigma}_{e}$ matrix
cannot be modifed on this screen.

\begin{figure}[H]
\noindent \begin{centering}
\includegraphics[width=6in]{Figures/SigmaError}
\par\end{centering}

\caption{Error Covariance \label{fig:Error-Covariance}}
\end{figure}



\subsubsection{Outcomes Covariance}

For designs with a Gaussian covariate, the user must specify the covariance
of the outcomes, $\boldsymbol{\Sigma}_{Y}$. For each independent
sampling unit, $\boldsymbol{\Sigma}_{Y}$ is the covariance of the
outcomes conditional on the fixed predictors. One can think of $\boldsymbol{\Sigma}_{Y}$
as the error covariance for each independent sampling unit in a model
containing only the fixed predictors and excluding the Gaussian covariate.
The \emph{Outcomes Covariance} screen allows the user to define $\boldsymbol{\Sigma}_{Y}$
by directly entering the covariance matrix values, as shown in Figure
\ref{fig:Outcomes-Covariance}. To ensure conformance with the $\boldsymbol{B}$
and $\boldsymbol{U}$ matrices, the dimensions of the $\boldsymbol{\Sigma}_{Y}$
matrix cannot be modifed on this screen.

\begin{figure}[H]
\noindent \begin{centering}
\includegraphics[width=6in]{Figures/SigmaY}
\par\end{centering}

\caption{Outcomes Covariance \label{fig:Outcomes-Covariance}}
\end{figure}



\subsubsection{Variance of Covariate}

For designs with a Gaussian covariate, the covariate is assumed to
have a Gaussian distribution with mean zero and variance $\sigma_{g}^{2}$.
The \emph{Variance of Covariate} screen allows the user to enter a
value for $\sigma_{g}^{2}$, as shown in Figure \ref{fig:Variance-of-Covariate}.

\begin{figure}[H]
\noindent \begin{centering}
\includegraphics[width=6in]{Figures/SigmaG}
\par\end{centering}

\caption{Variance of Covariate \label{fig:Variance-of-Covariate}}
\end{figure}



\subsubsection{Covariance of Outcomes and Covariate}

When controlling for a Gaussian covariate, power is typically improved
when the covariate explains some portion of the variance in the outcome.
The covariance matrix between the outcomes and the Gaussian covariate,
$\boldsymbol{\Sigma}_{YG}$, describes the association between the
outcomes and the Gaussian covariate. The \emph{Covariance of Outcomes
and Covariate} screen allows the user to specify values for $\boldsymbol{\Sigma}_{YG}$,
as shown in Figure \ref{fig:Covariance-of-Outcomes}. To ensure conformance
with the $\boldsymbol{\Sigma}_{Y}$ matrix, the dimensions of the
$\boldsymbol{\Sigma}_{YG}$ matrix cannot be modifed on this screen.

\begin{figure}[H]
\noindent \begin{centering}
\includegraphics[width=6in]{Figures/SigmaYG}
\par\end{centering}

\caption{Covariance of Outcomes and Covariate \label{fig:Covariance-of-Outcomes}}
\end{figure}



\subsubsection{Sigma Scale Factors}

GLIMMPSE allows users to specify scale factors for the covariance
matrices. For the general linear multivariate model with fixed predictors,
the scale factors are applied to the user-specified $\boldsymbol{\Sigma}_{e}$
matrix. For the general linear multivariate model with fixed predictors
and a Gaussian covariate, the scale factors are applied to the $\boldsymbol{\Sigma}_{e}$
matrix, which is calculated from $\boldsymbol{\Sigma}_{Y}$, $\boldsymbol{\Sigma}_{g}$,
and $\boldsymbol{\Sigma}_{Yg}$. Since variability can dramatically
impact power, it is common to calculate power for the proposed value,
as well as alternative values such as half and twice the proposed
value.

To specify one or more covariance scale factors, enter the scale factors
in the $\Sigma_{E}$\textit{ Matrix Scale Factors} box. After each
entry,press \includegraphics[height=11pt]{\lyxdot \lyxdot /2\lyxdot 0\lyxdot x/Figures/EnterKey}
on the keyboard. To delete a value, select the unwanted value and
click \includegraphics[height=11pt]{Figures/DeleteButton} to remove
the value from the list, as shown in Figure \ref{fig:Covariance-Scale-Factors-MatrixMode}.

\begin{figure}[H]
\noindent \begin{centering}
\includegraphics[width=6in]{Figures/SigmaScale}
\par\end{centering}

\caption{Covariance Scale Factors \label{fig:Covariance-Scale-Factors-MatrixMode}}
\end{figure}



\subsection{Options}

The screen shown in Figure \ref{fig:Options-MatrixMode} provides
an introduction to the Options section.

\begin{figure}[H]
\noindent \begin{centering}
\includegraphics[width=6in]{Figures/MatrixOptions}
\par\end{centering}

\caption{Options \label{fig:Options-MatrixMode}}
\end{figure}



\subsubsection{Power Calculation Method}

For designs with a baseline covariate, two different methods are available
to calculate power: quantile and unconditional power. For theoretical
details, please see \citet{glueck_adjusting_2003}. Select the power
methods by clicking the checkboxes shown in Figure \ref{fig:Power-Method-MatrixMode}.
If quantile power is selected, the user must also specify one or more
quantile values. For example, median power would be obtained by selecting
\emph{Quantile} and entering \textit{0.5} in the \textit{Quantiles}
list box. 

\begin{figure}[H]
\noindent \begin{centering}
\includegraphics[width=6in]{Figures/MatrixPowerMethod}
\par\end{centering}

\caption{Power Method \label{fig:Power-Method-MatrixMode}}
\end{figure}



\subsubsection{Confidence Intervals}

Power analysis involves some uncertainty in the choices for means
and variability. Therefore, the \emph{Confidence Intervals} screen
allows the user to request confidence intervals on the power results.
To include confidence intervals, uncheck the checkbox. The information
on the confidence interval screen, shown in Figure \ref{fig:Confidence-Intervals-MatrixMode},
describes the data set (or publication) from which the choices for
means and variances were obtained. For example, if a scientist was
calculating power based on the means and variances obtained from pilot
data, the scientist would enter information describing the pilot data
set. The following information is required:

The \emph{Assumptions} section allows the user to indicate if he or
she is uncertain about the variance, but reasonably certain of the
mean values, or uncertain of both the means and variance. 

The \emph{Upper and lower tail probabilities} define the width of
the confidence interval. For example, a centered 95\% confidence interval
would have both upper and lower tail probabilities of 0.025.

The \emph{Total sample size} value indicates the number of independent
sampling units in the pilot data set (or publication).

The \emph{Rank of the design matrix} describes a property of the predictor
matrix used in the pilot data set. Please see \citet{muller_linear_2006}
for details about matrix rank.

\begin{figure}[H]
\noindent \begin{centering}
\includegraphics[width=6in]{Figures/MatrixConfidenceIntervals}
\par\end{centering}

\caption{Confidence Intervals \label{fig:Confidence-Intervals-MatrixMode}}
\end{figure}



\subsubsection{Power Curve Options}

The \emph{Power Curve Options} screen allows the user to create a
power curve. A power curve describes the change in power (Y axis of
the power curve) relative to the total sample size, regression coefficient
scale factor, or the variability scale factor (all options for the
X axis of the power curve).

To create a power curve, the user must 1) uncheck the check box, 2)
select the value to appear on the horizontal axis, and 3) add one
or more data series, as shown in Figure \ref{fig:Power-Curve-MatrixMode}.

Depending on the study design, the user may request a large number
of power or sample size values in a single request. A data series
is defined by selecting a subset of the power or sample size values.
The user creates a data series by selecting values for several study
design variables and clicking the appropriate checkboxes. A data series
will be displayed as a single line on the power curve plot.

\begin{figure}[H]
\noindent \begin{centering}
\includegraphics[width=6in]{Figures/PowerCurveMatrixMode02}
\par\end{centering}

\caption{Power Curve \label{fig:Power-Curve-MatrixMode}}
\end{figure}


Note that the \emph{Power Curve Options} screen is the final screen
in the GLIMMPSE wizard. \textcolor{black}{If the study design is not
complete, the Calculate button will be disabled \includegraphics[height=11pt]{Figures/CalculateButtonDisabled}.}


\subsection{Calculate}

When sufficient information has been entered for your power or sample
size calculation, the \textit{Calculate} button will be highlighted
green. Click \includegraphics[height=11pt]{Figures/CalculateButtonEnabled}
to receive the results of your power analysis. Example results are
shown in Figure \ref{fig:Results-Table-MatrixMode}. For detailed
information regarding the Power Results table, refer to Table \ref{table:info}.
The resulting power curve is shown in Figure 

\begin{figure}[H]
\noindent \begin{centering}
\includegraphics[scale=0.6]{Figures/MatrixResults}
\par\end{centering}

\caption{Results Table\label{fig:Results-Table-MatrixMode}}
\end{figure}


\begin{figure}[H]
\noindent \begin{centering}
\includegraphics[scale=0.6]{Figures/PowerCurvePlotMatrixMode}
\par\end{centering}

\caption{Results Plot \label{fig:Results-Plot-MatrixMode} }


\end{figure}



\section{Additional GLIMMPSE Resources}

Additional resources for GLIMMPSE are available at \url{http://samplesizeshop.org}.
The Sample Size Shop project is a collaborative effort between the
University of Florida and the University of Colorado Denver. The goals
of the project are to develop new statistical methods for calculating
power and sample size, provide user friendly software to perform the
power and sample size calculations, and educate researchers regarding
both the methods and the software. The following online resources
are available for the GLIMMPSE software.
\begin{enumerate}
\item Tutorials demonstrating power and sample size calculations with GLIMMPSE
for a variety of study designs are available at \url{http://samplesizeshop.org/education/tutorials}
\item Validation reports showing the accuracy of GLIMMPSE calculations are
available at \linebreak\url{http://samplesizeshop.org/documentation/glimmpse/glimmpse-validation-results-2/}
\item Technical documentation for the software is available at \linebreak\url{http://samplesizeshop.org/documentation/glimmpse/}
\item GLIMMPSE software modules are available for downloaded from \linebreak\url{http://samplesizeshop.org/software-downloads/glimmpse-software-downloads/}
\end{enumerate}
\bibliographystyle{\string"C:/Users/munjala/Dropbox/Latex (1)/Local Tex Files/bibtex/bst/jss/jss\string"}
\addcontentsline{toc}{section}{\refname}\bibliography{GLIMMPSEBibliography}

\end{document}
